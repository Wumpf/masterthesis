\documentclass[thesis.tex]{subfiles}
\begin{document}

\chapter{Conclusion and Future Work}\label{chap:concl}


\section{Evaluation}
% HARD CRITERIA
% TODO? -> single indirect bounce, shadow + gloss
We have presented a novel real-time global illumination approach that works on arbitrary scenes without any pre-computations.
Similar to Vardis et al. \cite{bib:radiancecachechromaticcompression} we use a regular grid of light caches that are activated on demand.
\\
% -> no pre-computation
This makes the technique entirely invariant of animations and allows to perform arbitrary scene changes at runtime.
The same goes for materials which are allowed to have arbitrary variations over surfaces (contrary to some radiance caching techniques like LightSkin \cite{bib:LightskinPaper}).
\\
% -> gpu only
All steps are executed on the GPU without any transfers to the CPU or back, except camera and scene settings.
This way the approach is free of any synchronization stalls and frees the CPU for other tasks.

% SOFT CRITERIA
% -> temp coherency
We listed temporal coherency as most important quality criterion.
While our algorithm is rather stable, we were not able to avoid this issue entirely:
Typical reflective shadow map flickering can occur when directly lit objects move.
Another source of incoherency are high resolution specular environment map.
Note however, that there are no incoherencies at all for camera movements in general and animated objects that are receive little direct light.

% -> quality
As far as other artifacts are concerned, our approach fares relatively well.
While it suffers from light bleeding like most techniques, our cascaded address volume is able to limit these problems adaptively.
Using simplified voxel cone tracing we were able to create soft and convincing indirect shadows.
Indirect specular lighting is quality-wise clearly better solved in a few other techniques like voxel cone tracing \cite{bib:voxelconetracing}.
Still, our new idea of hemispherical environment maps achieves better results than most competing methods with similar constraints even at low resolutions.
\autoref{sec:eva:errorsources} provided a concise analysis of all error sources when compared to a ground-truth solution within the given light path types.

% -> performance
Contrary to previous work our approach selects relevant caches in screen space and stores them in a continuous buffer.
This allows us to compute indirect lighting very efficiently.
However, as we traverse every single reflective shadow map pixel (similar to in LightSkin \cite{bib:LightskinPaper} but unlike to most techniques which only sample the RSM) this pass is still very expensive.
In this context the simplifications introduced by the shadow lod showed to be vital for the overall performance.
In comparison to the actual cache lighting, the addressing and interpolation overhead seems to be rather low.
Because of this we refrained to use down-sampling for our screen-depended passes.
As all cache-based methods, the technique is relatively invariant to resolution changes and is ready for modern "ultra  high definition" displays without the need of upsampling.
In general our approach can performance-wise compete with other techniques on the field as long as indirect specular lighting is not activated.
The later has unfortunately, depending on the RSM resolution, a seemingly disproportionate impact on the duration of the cache lighting pass.

% -> memory
By using modern GPU programming strategies, we were able to work with a very limited memory budget which is superior to the consumptions of most techniques.
Again, the specular lighting has the highest impact, followed by the RSM which may be rendered with a lower resolution (loosing the advantages of down-sampling and sharing of shadow map for direct lighting).

% -> scale
By cascading our address volumes we are able to handle large scenes easily.
Though, the large gaps between caches in higher/distant cascades may lead to artifacts, especially for indirect shadow.
It may be advisable to perform shadowing and specular lighting only for caches in the closer cascades.



\section{Future Work}

Solid Voxelization: "Fast Parallel Surface and Solid Voxelization on GPUs"  Schwarz \& Seidel 


* something temporal? accumulating information?
* Alter lookup lookup direction for caches to improve specular sampling
  * Similar to SH gradients
  * It should somehow be possible to...
* "supergrid" shares certain values between caches
* better fill hole: dilate on lowest (or all?) level intead of pull/push like approach ?
* only relevant parts on specular envmap
* Jitter cache grid
  * tried shortly, but looked like a bad idea
* different projection for specenvmap
* do chroma downsampling as in \cite{bib:radiancecachechromaticcompression}

* adaptive shadow lod depending on expected cone size

\subfilebib % Makes bibliography available when compiling as subfile
\end{document}