\documentclass[thesis.tex]{subfiles}
\begin{document}

\appendix

\chapter{Spherical Harmonics Evaluation} 
All vectors in this appendix are given in spherical coordinates.
To convert to Cartesian coordinates see \autoref{equ:postoangle}.

\section{Polynomial Forms of Spherical Harmonics Basis}
\begin{align}
y^0_0(\theta, \varphi) =& \frac{1}{2\cdot \sqrt{\pi }} \\
y^{-1}_1(\theta, \varphi) =& \frac{-\sqrt{3}\cdot \mathrm{sin}\left( \theta\right) \cdot \mathrm{sin}\left( \varphi\right) }{2 \sqrt{\pi }} \\
y^0_1(\theta, \varphi) =& \frac{\sqrt{3}\cdot \mathrm{cos}\left( \theta\right) }{2 \sqrt{\pi }} \\
y^1_1(\theta, \varphi) =& \frac{-\sqrt{3}\cdot \mathrm{sin}\left( \theta\right) \cdot \mathrm{cos}\left( \varphi\right) }{2 \sqrt{\pi }} \\
y^{-2}_2(\theta, \varphi) =& \frac{\sqrt{15}\cdot \mathrm{sin}\left( \theta\right) \cdot \mathrm{sin}\left( \varphi\right) \cdot \mathrm{sin}\left( \theta\right) \cdot \mathrm{cos}\left( \varphi\right) }{2 \sqrt{\pi }} \\
y^{-1}_2(\theta, \varphi) =& \frac{-\sqrt{15}\cdot \mathrm{sin}\left( \theta\right) \cdot \mathrm{sin}\left( \varphi\right) \cdot \mathrm{cos}\left( \theta\right) }{2 \sqrt{\pi }} \\
y^0_2(\theta, \varphi) =& \frac{\sqrt{5}\cdot \left( 3\cdot {{\mathrm{cos}\left( \theta\right) }^{2}}-1\right) }{4 \sqrt{\pi }} \\
y^{1}_2(\theta, \varphi) =& \frac{-\sqrt{15}\cdot \mathrm{sin}\left( \theta\right) \cdot \mathrm{cos}\left( \varphi\right) \cdot \mathrm{cos}\left( \theta\right) }{2 \sqrt{\pi }} \\
y^{2}_2(\theta, \varphi) =& \frac{\sqrt{15}\cdot \left( {{\left( \mathrm{sin}\left( \theta\right) \cdot \mathrm{cos}\left( \varphi\right) \right) }^{2}}-{{\left( \mathrm{sin}\left( \theta\right) \cdot \mathrm{sin}\left( \varphi\right) \right) }^{2}}\right) }{4 \sqrt{\pi }}
\end{align}

\newpage

\section{Clamped Cosine Lobe in Spherical Harmonics Representation} \label{chap:shcosinelobe}
Spherical Harmonics coefficients for a clamped cosine lobe in a given direction.

\begin{align}
c^0_0 =& \frac{\sqrt{\pi }}{2}\\
c^{-1}_1 =& -\frac{\sqrt{\pi }\cdot \mathrm{sin}\left( \varphi\right) \cdot \mathrm{sin}\left( \theta\right) }{\sqrt{3}}\\
c^0_1 =& \frac{\sqrt{\pi }\cdot \mathrm{cos}\left( \theta\right) }{\sqrt{3}}\\
c^1_1 =& -\frac{\sqrt{\pi }\cdot \mathrm{cos}\left( \varphi\right) \cdot \mathrm{sin}\left( \theta\right) }{\sqrt{3}}\\
c^{-2}_2 =& \frac{\sqrt{15 \, \pi}\cdot \mathrm{cos}\left( \varphi\right) \cdot \mathrm{sin}\left( \varphi\right) \cdot {{\mathrm{sin}\left( \theta\right) }^{2}}}{8}\\
c^{-1}_2 =& -\frac{\sqrt{15 \, \pi}\cdot \mathrm{sin}\left( \varphi\right) \cdot \mathrm{cos}\left( \theta\right) \cdot \mathrm{sin}\left( \theta\right) }{8}\\
c^0_2 =& \frac{\sqrt{5 \, \pi}\cdot \left( 3\cdot {{\mathrm{cos}\left( \theta\right) }^{2}}-1\right) }{16}\\
c^1_2 =& -\frac{\sqrt{15 \, \pi}\cdot \mathrm{cos}\left( \varphi\right) \cdot \mathrm{cos}\left( \theta\right) \cdot \mathrm{sin}\left( \theta\right) }{8}\\
c^2_2 =& \frac{\sqrt{15 \, \pi}\cdot \left( {{\mathrm{cos}\left( \varphi\right) }^{2}}\cdot {{\mathrm{sin}\left( \theta\right) }^{2}}-{{\mathrm{sin}\left( \varphi\right) }^{2}}\cdot {{\mathrm{sin}\left( \theta\right) }^{2}}\right) }{16}
\end{align}


\end{document}