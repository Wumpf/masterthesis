\documentclass[thesis.tex]{subfiles}
\begin{document}

\chapter{Related Work}
\label{chap:prevwork}

% Categories from GISTAR
% * Finite Elements (also contains voxel based GI)
% * Monte-Carlo ray tracing
% * Photon Mapping
% * Instant Radiosity
% * Many lights (hierarchical! gathering of lights)
% * Point-based (render points somehow)
% * Discrete ordinate methods (propagation volumes etc.)
% * Precomputation

Awesome overview of all realtime global illumination techniques before 2012 \cite{bib:RealtimeGIOverview} Ritschel et al..

\section {Instant Radiosity \& Many Lights Methods}
Survey about many lights in general \cite{bib:manylightssurvey2014}.


\subsection{Reflective Shadow Mapping}
Dachsbacher et. al. \cite{bib:reflectiveshadowmaps}

Various Clustering attempts.

Problems with VPLs. Use area lights.

\subsection{Shading}
Multi-resolution splatting techniques ([All Nichols] Hierarchical
image-space radiosity for interactive global illumination, Multiresolution splatting for indirect illumination, Interactive indirect illumination using adaptive multiresolution splatting) are not as efficient as "Tiled Deferred Shading" (according to \cite{bib:clusturedpreconvoledradiancecaching} (where it is just a side note!)) ... logically Tiled Deferred Shading is even better.
Interleaved Sampling (Interleaved sampling. In
Proc. of Eurographics Workshop on Rendering (2001)) also a good idea 

Clustured Forward/Deferred Shading: Finding Unique clusturs is DIFFERENT in paper \cite{bib:clusturedshading} and practical implementation like "Pratical Clustured Shading" (SIGG2013, Humus). Paper uses (local) sorting and page tables. Humus uses Volume Texture. Ollson Siggra2013 presentation uses parallel prefix sum.

\subsection{Shadows for Many Lights}
Book about realtime shadows (2011)\cite{bib:realtimeshadowsbook}.\\
Imperfect shadow maps Ritschel et. al. \cite{bib:imperfectshadowmaps}.\\

Virtual Shadows: Shadow Mapping using a Virtual Texturing. TODO Read, there is much more to this. \cite{bib:virtualshadowmaps}

\section{Discrete Ordinate Methods}
GPU Radiosity (because its the king of discrete methods!).\\
Light Propagation Volumes.\\
Voxel Cone Tracing and Voxel Based GI.\\
Radiance Hints \cite{bib:radiancehints}.

\section{Light Skin}
There is this paper \cite{bib:LightskinPaper} of Lensing et. al.

\section{(Ir-)Radiance Caching}
Irradiance Caching \cite{bib:irradiancecaching}.\\
Radiance Caching \cite{bib:radiancecaching}.\\
GPU versions.


(Clustered) Preconvoled Radiance Caching \cite{bib:clusteredpreconvoledradiancecaching}
"Pre-convolved radiance caching (PCRC) [SNRS12] proposes to store incident radiance in a texture with one texel per direction and preconvolve it—in essence creating a mip-map-pyramid of the
radiance texture with accumulated radiance in the higher
levels. Additionally, a low-resolution mip-map level is cosine folded to store irradiance. Using these two textures, evaluating the RCs requires only two texture lookups instead of re-evaluating the reflection functions. Irradiance can also be gathered into screen-space caches from a photon map [WWZ09] to integrate caustics." (from \cite{bib:clusteredpreconvoledradiancecaching})\\
Improvements in the clustured algorithm: "Creation" via Voxel Cones, "Distribution" in ScreenSpace, "Evaluation" gathering instead of splatting. The "Distribution" works similar to clustured deferred shading - each cluster results in a radiance cache with an interesting way to extract the normal. Evaluation scheme is extremely simple, does not take cache normals into account. Of course no temporal coherence -.-

\section{Enlighten}?

\section{Voxel Cone Tracing}

\todo{original voxel cone tracing} Has been evaluated in game engines. "MITTRING M.: The Technology Behind the Unreal Engine 4 Elemental demo. SIGGRAPH 2012 Advances in RealTime Rendering in 3D Graphics and Games Course, 2012." 

Layered Reflective Shadow Maps for Voxel-based Indirect Illumination decouples occlusion from lighting data.
Visibility determination is handled via Voxel Cone tracing while the actual lighting is performed by lookups in a pre-filtered Layered Reflective Shadow Map.
Doing so avoids most of the memory and initialization/update overhead associated with the large voxel data structure, since each voxel encodes only binary visibility information.
Layers to avoid discontinuities, etc.


\section{Voxelization}

\cite{bib:GPUGems2}[Chapter 42] $http://http.developer.nvidia.com/GPUGems2/gpugems2_chapter42.html$

Modern GPU in OpenGL Insights \cite{bib:openglinsightsvoxel}

Maxwell GPUs per Hardware

\subfilebib % Makes bibliography available when compiling as subfile
\end{document}