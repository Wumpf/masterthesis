\documentclass[thesis.tex]{subfiles}
\begin{document}

\chapter{Main}\label{chap:basics}

\section{Caches}
\subsection{Cache Generation}
Cache properties (on creation):
\begin{easylist}[itemize]
# position (world space)
# normal
# material properties? (roughness etc.)
\end{easylist}

Goals:
\begin{easylist}[itemize]
# Avoids precomputation
# Temporal coherency
# roughly equal distribution in screen space
## Less caches in distant regions, more caches in near regions
\end{easylist}
Approaches:
\begin{easylist}[itemize]
# GBuffer extract
## super slow since many neighboring pixels use atomics to write into same memory locations
## temporarly relatively smooth under scene changes, but visible information loss and changes at depth discontinuities

# During (converative) Voxelization
## performance ok'ish
## Bad coherence under scene changes if voxel resolution not high enough. Many sources:
### triangles that add their sample to a voxel are changing from one frame to another
### independently the same happens to the number of caches available for interpolation (might also be a problem in previous GBuffer extract approach)
### Depending on rotation, the normal average can change rather fast

\end{easylist}

Note on Voxelization: Voxelization needs to be conservative, otherwise holes. Neighborsearch can not fill those!

\subsection{Memory representation}
Non-perfect Hashing. Striving for low collision count.
Still much better than full grid!

\subfilebib % Makes bibliography available when compiling as subfile
\end{document}