\documentclass[thesis.tex]{subfiles}
\begin{document}

\chapter{Main}\label{chap:basics}

\section{Overview}

\section{Realtime Light Cache Placement}

\subsection{Abandoned Approaches}

\subsection{ACTUAL TECHNIQUE NAME}

Cache properties (on creation):
\begin{easylist}[itemize]
# position (world space)
# normal
# material properties? (roughness etc.)
\end{easylist}

Goals:
\begin{easylist}[itemize]
# Avoids precomputation
# Temporal coherency
# roughly equal distribution in screen space
## Less caches in distant regions, more caches in near regions
\end{easylist}
Approaches:
\begin{easylist}[itemize]
# GBuffer extract
## super slow since many neighboring pixels use atomics to write into same memory locations
## temporarly relatively smooth under scene changes, but visible information loss and changes at depth discontinuities

# During (converative) Voxelization
## performance ok'ish
## Bad coherence under scene changes if voxel resolution not high enough. Many sources:
### triangles that add their sample to a voxel are changing from one frame to another
### independently the same happens to the number of caches available for interpolation (might also be a problem in previous GBuffer extract approach)
### Depending on rotation, the normal average can change rather fast
\end{easylist}

Note on Voxelization: Voxelization needs to be conservative, otherwise holes. Neighborsearch can not fill those!

\subsection{Memory representation}
Non-perfect Hashing. Striving for low collision count.
Still much better than full grid!

\section{Cache Lighting}

\subsection{Shadowing}

\section{Cache Interpolation}



\section{Implementation Details}

As many as possible details should be delayed into this chapter. If it gets large or starts to mirror the main part, make it a chapter!

This is the right place for describing how to use Compute, OpenGL etc. for achieving the rather abstract formulated goals of the sections before.

Deferred Renderer, 32bits per Layer RGB(A) srgb - Diffuse, RG 16snorm - Normals with angles, extra infos todo, R32F Depth Buffer (swapped near/far)\\
(This detail belongs more or less to Eva...)

\subfilebib % Makes bibliography available when compiling as subfile
\end{document}